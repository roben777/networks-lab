\documentclass[12pt]{book}              
%\parindent0pt  \parskip10pt             % make block paragraphs
%\raggedright                            % do not right justify

\usepackage[margin=2cm]{geometry}
\usepackage{enumitem}
%\usepackage{url}
\usepackage{siunitx}
\usepackage{listings}
\usepackage{graphicx} 
\usepackage{booktabs} % for much better looking tables
\usepackage{array} % for better arrays (eg matrices) in maths
\usepackage{verbatim} % adds environment for commenting out blocks of text & for
\usepackage{subcaption} % make it possible to include more than one captioned fi
\usepackage{float}
\usepackage[hyperfootnotes=false]{hyperref}
%\floatstyle{boxed} 
\restylefloat{figure}
\usepackage{tabularx}

\lstset{
%  framexleftmargin=10mm, 
%  tabsize=2,
%  language=c, 
  frame=single,
  numbers=none,
  keepspaces,
%  basicstyle=\footnotesize,
  basicstyle=\scriptsize\ttfamily,
}

\usepackage{tcolorbox}
\usepackage{xcolor}

\usepackage[
    type={CC},
    modifier={by-sa},
    version={4.0},
]{doclicense}


\title{Computer Networks - Laboratory manual}    % Supply information
\author{Robert Benkoczi, University of Lethbridge}    %   for the title page.
\date{\today\\ version 2.0}  %   Use current date. 

%%%%%% macros %%%%%%5
\newcommand{\kathara}{Kathar\'a}

%%% inserting figures 
%%% arg 1: figure file (no directory)
%%% arg 2: scale factor
%%% arg 3: caption
%%% labels are #1.fig
\newcommand{\insfig}[3]{
\begin{figure}[tb]
\begin{center}
\includegraphics[scale=#2]{figs-manual/#1}
\caption{#3\label{#1.fig}}
\end{center}
\end{figure}
}


%\newenvironment{console}
%               {\begin{scriptsize} \begin{verbatim}}{\end{verbatim}  \end{scriptsize}}

%%%%%%%%%%%%%%%%%%%%%%%%

%\renewcommand{\chaptername}{Activity}

% Note that book class by default is formatted to be printed back-to-back.
\begin{document}                        % End of preamble, start of text.
\frontmatter                            % only in book class (roman page #s)
\maketitle    
\doclicenseThis                      
\tableofcontents                        % Print table of contents

\mainmatter % arabic page nrs

\chapter{Preparing to run the labs at home}\label{intro.se}

The laboratory experiences in this document use a network emulator called \kathara available at \url{kathara.org}. \kathara is installed in CS labs at the univesity. You may choose to install \kathara\ on your own computer and you can complete the exercises at home, if you wish. \kathara\ is cross platform, although you need a Pro version for the Windows operating system since \kathara\ uses \emph{Docker} (\url{https://www.docker.com/}).

\section{Installing and configuring \emph{\kathara} at home}

This section applies only if you wish to install \kathara\ at home. In the CS laboratoory at the university, \kathara\ is already installed.

\begin{tcolorbox}[title=Lab safety note]
  Please exercise common sense when attempting any
  of the commands described in this manual on real computer
  networks. Please consult the UofL policies concerning the use of the
  data network \cite{its-policies} and also be aware of the federal
  laws that govern the use of computers and data networks
  \cite{law-canada,law-us}. 
\end{tcolorbox}

\paragraph{Installing \kathara}

\begin{enumerate}[label=(\arabic*)]
\item You will need root access or admin privileges on
  the computer you are installing \kathara\ on. Please do not execute
  this step on the school network.

\item Download the installation files for \kathara\ from
  \url{https://github.com/KatharaFramework/Kathara/releases}. 
Download both the \verb$KatharaGUI$ package for your platform and the
source code package. The source code package contains additional
folders, for example the \verb$bin$ folder, needed to run the
emulator. 

\item Unpack both package files into a system folder, such as
\verb$/opt/kathara$. 

\item Follow the installation instructions from
\url{https://github.com/KatharaFramework/Kathara/wiki/Linux} for a
Linux installation or the appropriate instructions for your platform
available from the wiki page \cite{kathara-wiki}. 

\end{enumerate}


%%%%%%%%%%%%%%%%%%%%%%%%%%55


\chapter{Lab 1: an introduction to the tools}

\section{Objectives}

In this lab, students will learn some basic commands and configuration options to get them started with \kathara. No networks are simulated yet. Some basic familiarity with linux shell commands is assumed. Students without experience with shell commands will learn during this lab.

\begin{enumerate}[label=Objective \arabic*:]
  \item\label{vstart} Students will start a single device (docker container) with \kathara\ and will execute unix commands including \emph{ifconfig} and \emph{ping}. This tests if their \kathara\ environment is correctly setup.
  \item\label{config} Students will adjust the configuration of their \kathara\ system by changing configuration files.
  \item\label{hosthome} Students will transfer information between their virtual device and the host machine. Students will create files on their virtual devices that can be visible on their host machine, and vice-versa.
  \item\label{image} Students will start and stop the execution of devices using different docker images for \kathara.
    \item\label{bridged} Students will start a virtual device running the \emph{kathara/base} docker image and configured with one network interface linked to their host machine. Students will use test this network link with \emph{ping(6)}. 
\end{enumerate}


\subsection{Summary of commands}

\begin{tabularx}{\textwidth}{r  X}
  Commands & Description \\ \midrule
  ifconfig & lists and configures network interfaces \\
  ip & show / manipulate routing, devices, policy routing and tunnels \\
  ping & tests network connectivity to a remote or to the same computer \\
  nc (netcat) & network cat: transfers data through the network to a remote computer \\ \midrule

  ls & lists the contents of the current directory \\
  cd & changes the current directory \\
  cat & conCATenates the content of the files passed as arguments or takes input from the terminal if no arguments are given \\
  export & exports an environment variable so that it is defined for the current terminal session \\
  whoami & prints the account name for the terminal session\\
  vim, nano & start text editors on devices \\ \midrule
  vstart & starts a \kathara\ virtual device; \emph{-n name} is mandatory \\
  vclean & stops a virtual device;  \emph{-n name} is mandatory \\
  list & lists all running virtual devices \\
  wipe & shuts down all \kathara\ virtual devices\\
\end{tabularx}



\section{Lab activities}

\paragraph{Target: \ref{vstart}}

\begin{enumerate}[label=Activity \arabic*:]
%%   \item To be able to start a device (virtual machine) successfully, as a first step, we need to define an environment variable that points to the install location for the \kathara\ scripts (step 6 in
%% the Linux installation instructions). In your school network, the kathara environment seems to be properly configured; executing this step may not be necessary. 
    
%% \begin{verbatim}
%% host$ export NETKIT_HOME=/opt/kathara/bin/
%% \end{verbatim}

  %%
\item Follow the instructions contained in the document ``Start \kathara\ at school'' under the Laboratory section on Moodle.
  
\item Start your first virtual device (docker container). Give your device a name. The name will appear in the title bar of the terminal opened for that device. In the example below, the name of the device is ``pc''.

  \begin{lstlisting}
host$ kathara vstart -n pc
====================== Starting Device ======================
Deploying devices... |##################################################################| 1/1
\end{lstlisting}
%$
  
\item Your device has started successfully when a new terminal window opens. This terminal window allows to run commands on the virtual device. Try to navigate through the file system for your device. What is the user name for the terminal on the virtual device? Can you see the same set of files in the device window as in a regular terminal window opened on the host machine?

\begin{lstlisting}
root@pc:/# ls
bin   dev  home      lib    media  opt   root  sbin    srv  tmp  var
boot  etc  hosthome  lib64  mnt    proc  run   shared  sys  usr
root@pc:/# cd 
root@pc:~# ls
root@pc:~# pwd
/root
root@pc:~# whoami
root
root@pc:~# 
\end{lstlisting}

\item A network interface is an operating system gadget that allows users interact and configure the behaviour of the OS portion of the OS that deals with the network. To see your network interfaces, run \emph{ifconfig}. Another command that shows the available network interfaces and their assigned IP addresses is \emph{ip address}. What do you notice if you run ifconfig on the virtual device versus when you run it on the host computer? NOTE: \emph{ifconfig} is not available on the kathara host machines in the lab, use \emph{ip address}.

\begin{lstlisting}
root@pc:~# ifconfig
lo: flags=73<UP,LOOPBACK,RUNNING>  mtu 65536
        inet 127.0.0.1  netmask 255.0.0.0
        inet6 ::1  prefixlen 128  scopeid 0x10<host>
        loop  txqueuelen 1000  (Local Loopback)
        RX packets 0  bytes 0 (0.0 B)
        RX errors 0  dropped 0  overruns 0  frame 0
        TX packets 0  bytes 0 (0.0 B)
        TX errors 0  dropped 0 overruns 0  carrier 0  collisions 0
\end{lstlisting}

\item The only network interface available to your virtual device started as illustrated in this lab is \emph{lo}. Find out what this network interface is about.

\item \emph{ping} is a program that sends a sequence of control packets to the address specified as an argument. It is a standard tool to verify the connectivity between the computer on which ping is executed and the remote computer specified as an argument. The loopback network interface is always available on a device and allows connectivity with the device itself. The name \emph{localhost} identifies the device itself as an address (name actually). We can test the connectivity with the device itself.

\begin{lstlisting}
root@pc:/# ping localhost
PING localhost(localhost (::1)) 56 data bytes
64 bytes from localhost (::1): icmp_seq=1 ttl=64 time=0.019 ms
64 bytes from localhost (::1): icmp_seq=2 ttl=64 time=0.060 ms
64 bytes from localhost (::1): icmp_seq=3 ttl=64 time=0.067 ms
^C
--- localhost ping statistics ---
3 packets transmitted, 3 received, 0% packet loss, time 2028ms
rtt min/avg/max/mdev = 0.019/0.048/0.067/0.022 ms
root@pc:/# 
\end{lstlisting}

\item What inet (IPv4) address is asssociated with the local host? How about the inet6 (IPv6) address for the local host? Hint: examine the output of the \emph{ifconfig} command. Run \emph{ping} with these addresses instead of the \emph{localhost} name.

  Note that IPv6 is not initially enabled on the virtual devices. To enable IPv6, you will need to edit kathara's configuration file on the host machine. This step is described shortly. Come back to this step after you enable IPv6 and start a new virtual machine.

\item You can shut down a virtual device you created with command \emph{vstart} by running command \emph{vclean}. Shutdown the device you started at the beginning of the lab.

\begin{lstlisting}
$ kathara vstart -n pc1
====================== Starting Device ======================
Deploying devices... |##############################################################################| 1/1
$ kathara vclean -n pc1
Deleting devices... |###############################################################################| 1/1
INFO - Device `pc1` deleted successfully!
\end{lstlisting}
\end{enumerate}

%%%%%%%%%%
\paragraph{Target: \ref{image}}

\begin{enumerate}[resume*]
\item The \emph{vstart} command has other options besides the mandatory argument \emph{-n}. Check \url{https://www.kathara.org/man-pages/kathara-vstart.1.html} for the man pages. In this activity, you will experiment with a few options for \emph{vstart}. In particular, we will look at the \emph{--image} option. The virtual device you started is actually a \emph{docker container} \cite{docker-useful}. A docker container is built from an \emph{image} which contains all the software needed for the docker application. When starting a virtual device, the default image used for your device is \emph{kathara/quagga}. This image contains packet routing software, \url{https://quagga.net/}. Some useful network tools, such as \emph{netcat} \cite{netcat}, are not available in the default image.

  Start a virtual device using the \emph{kathara/base} image. Note that the output might differ from the one illustrated below. The new image must be fetched from \url{https:\\hub.docker.com}, which happens automatically when you start the virtual machine.

\begin{lstlisting}
host$ kathara vstart -i kathara/base -n pc2
====================== Starting Device ======================
Deploying devices... |#######################################################################################| 1/1
\end{lstlisting}
%$

\item Go to \url{https://hub.docker.com} and search for images that contain the keywork \emph{kathara}. Start another virtual device using a different image from \emph{quagga} and \emph{base}. Then list your running devices with \emph{kathara list}.

  Note: in the CS lab, the \emph{list} command cannot be carried out because of permissions. You are encouraged to install \kathara\ on your own computer and experiment with the list command as well as docker command.

\begin{lstlisting}
$ kathara list
TIMESTAMP: 2021-09-04 17:02:43.932187


 LAB HASH                USER      DEVICE NAME  STATUS   CPU %  MEM USAGE / LIMIT   MEM %  NET I/O   

 BquVk2860DTFrej8slHuVg  rbenkocz  pc2          running  0.00%  1.85 MB / 15.31 GB  0.01%  0 B / 0 B 

 BquVk2860DTFrej8slHuVg  rbenkocz  pc3          running  0.00%  1.84 MB / 15.31 GB  0.01%  0 B / 0 B 

 BquVk2860DTFrej8slHuVg  rbenkocz  pc           running  0.00%  3.76 MB / 15.31 GB  0.02%  0 B / 0 B 

\end{lstlisting}

%$
\end{enumerate}

\paragraph{Target: \ref{config}}

\begin{enumerate}[resume*]
\item For many of the future laboratories, we would like to use the \emph{kathara/base} image. We can change the default behaviour of kathara by editing its configuration file. Open the \emph{./config/kathara.conf} file on the host machine and change the default image. \textbf{While we are here, also change the hosthome\_mount option to true and enable ipv6.}
\begin{lstlisting}
{
 "image": "kathara/base",
 "manager_type": "docker",
 "terminal": "/usr/bin/xterm",
 "open_terminals": true,
 "device_shell": "/bin/bash",
 "net_prefix": "kathara",
 "device_prefix": "kathara",
 "debug_level": "INFO",
 "print_startup_log": true,
 "enable_ipv6": true,
 "last_checked": 1630600319.5141602,
 "hosthome_mount": true,
 "shared_mount": true
}
\end{lstlisting}
\end{enumerate}

\paragraph{Target: \ref{hosthome}}

\begin{enumerate}[resume*]
\item One important feature of the network emulator is the ability of transfering information (for example network capture files) between the virtual devices and the host computer. On any virtual device, if the \kathara\ configuration option \verb$hosthome_mount$ is enabled, the host computer home directory is accessible to the virtual device via the directory \verb$/hosthome$ on the virtual device.

  \begin{itemize}[label=--]
  \item Create a working folder in your home directory \textbf{on the host computer} named \emph{work1}.
  \item Make sure the \kathara\ configuration option \emph{hosthome\_mount} is enabled.
  \item Make sure you have a virtual device started \emph{after} the hosthome mount option is enabled.
  \item In the terminal window of the virtual device, navigate to the directory you just created. Use the text editor \emph{nano} to create a text file with some line of text, such as ``Hello world.''.
  \item In the terminal window on the host machine, navigate to the \emph{work1} directory. List it with \emph{ls}. Does the file you created from the virtual device exist and is it accessible on the host machine? Which user owns the file (run \emph{ls -l})? Can you remove it from the host machine? (If not, you can remove it from the virtual device).
  \end{itemize}

\end{enumerate}

\paragraph{Target: \ref{bridged}}

\begin{enumerate}[resume*]
\item The virtual devices we started so far have only one network interface enabled, the loopback interface. It is possible to instruct \kathara\ to ``connect'' the virtual device to the host computer via a second network interface on the virtual device. This is achieved with command line option \emph{--bridged} with the \emph{vstart} command.

\begin{lstlisting}
$ kathara vstart -i kathara/base -n pc100 --bridged
====================== Starting Device ======================
INFO - Pulling image `kathara/base:latest`... This may take a while.
Deploying devices... |###############################################################################################| 1/1
$
\end{lstlisting}

\item Use the command \emph{ifconfig} on the virtual device to find out the IP address of the new network interface that connects your device with the host machine. Run \emph{ping} from the host machine to test the connectivity with the virtual device. 

\item When starting a virtual device that is \emph{bridged} with the host computer, a new interface is also created on the host machine. In the lab environment, the name of this interface starts with \emph{docker}. The host machines \emph{beta-kathara} and \emph{gamma-kathara} do not have \emph{ifconfig} installed. Use instead the command \emph{ip address} and identify:
\begin{itemize}[label=-]
\item The IP address of the network interface connected to the virtual device.
\item The real IP address of the host computer (the IP address used in the Internet).
\end{itemize}
Test with \emph{ping} both addresses FROM the virtual device. 
\end{enumerate}

\paragraph{Cleanup}

\begin{enumerate}[resume*]
\item Shutdown all the virtual devices you created during the lab. Use \emph{kathara wipe}.

\begin{lstlisting}
$ kathara wipe
Are you sure to wipe Kathara? (y/n) y
$ 
\end{lstlisting}
\end{enumerate}

\section{Exercises}

\begin{enumerate}
\item Start a virtual device named ``mypc'' using the \emph{kathara/base} image. Shut down the virtual device ``mypc''.
\item List the virtual devices started on your host machine (can be done on your personal installation of \kathara\ at home).
\item\label{redirect} Open a terminal window, which can be a terminal
  on the host machine or the terminal for a virtual machine. Run
  command \emph{ls $>$ listing.txt}, which redirects the output of command \emph{ls} to a file, which is, in this case, named \emph{listing.txt}. Examine the contents of the file using the commands \emph{cat} or \emph{less}. 

Check the manual page for command \emph{factor}. Run command \emph{factor} on an integer that has at least 6 digits and redirect the output to a text file named \emph{factors.txt} in your home directory. 

\item Start a virtual device. On the virtual device, run command \emph{ifconfig} and redirect its contents to a file in the work directory on your host machine. List the contents of the file from the host machine. You can use character $>$ followed by the file name to redirect the output of any command, from the console to the specified file (see Exercise \ref{redirect}).

\item Start a virtual device that is \emph{bridged} with the host computer. Find out the IPv6 addresses of the virtual device and the corresponding link local IPv6 address of the host machine. Use IPv6 with \emph{ping6} and perform the following tasks (remember to append the network interface name to link local IPv6 addresses):
\begin{enumerate}[label=(\alph*)]
\item Test the connectivity via IPv6 with \emph{ping6}, from the host machine to the virtual device.
\item Test the connectivity via IPv6 with \emph{ping6}, from the virtual device to the host machine.
\item Are you able to \emph{ping6} the host machine from the virtual device using the link local IPv6 address of the host machine attached to the real network interface? Explain.
\item Are you able to \emph{ping6} the virtual device from the host machine using the link local IPv6 address of the virtual device but with the real network interface of the host machine after the \% character? Explain.
\end{enumerate}

\item Start a virtual device that is \emph{bridged} with the host computer. Compile, on your HOST machine, a program capable of sending a UDP packet using IPv6 (or IPv4) (revisit your first two labs). Start \emph{netcat} on the virtual device in listen mode. Send a message using your program on the host machine to the \emph{netcat} process on the virtual device.
\end{enumerate}

%%%%%%%%%%%%%%%%%

\chapter{Lab 2: creating networked devices}



\begin{itemize}[label=--]
\item Students will define a simple local network with
  two hosts and
  will use the ping utility to test the connection between the hosts.

\item If the network is properly defined, the emulator will correctly
  configure IPv4 addresses for the two hosts. Using standard commands
  available for hosts running Unix based operating systems, the
  students will configure IPv6 addresses and will test the
  connectivity using the IPv6 network protocols.

%% \item Students will use the \emph{tcpdump} program to capture network
%%   packets. Students will examine captured packets with
%%   \emph{Wireshark} on their personal computers, or with an online
%%   network packet visualization tool such as
%%   \url{https://hpd.gasmi.net/}. Students will identify the protocols
%%   that participated in the creation of the captured packets. 
\end{itemize}


\section{Lab activities}


\subsection{Creating a first network}\label{2nodes.se}

To test \kathara\ and to begin the second part of this lab, we must
define the network to be simulated. This definition resides in a
collection of configuration files that contains the topology of the
network as well as configuration files loaded by the virtual nodes in
the network. We call these configuration files \emph{lab files}.

Lab files are most easily created using the KatharaGUI application,
strangely available under the systems folder you installed \kathara\ 
(for example \verb$/opt/kathara/$) 
or under \verb$/opt/kathara/GUI/$ in your CS network at school. 

The following fields should be filled:
\begin{description}
\item[Lab information:] it is a good idea to add a description, so you
  can distinguish the generated files later on. When submitting
  exercises for an assignment, be sure to enter your name in the
  author field.
\item[Machine informations:] Fill a name for the first machine
  (host). It is a good idea to provide unique names to the hosts you
  create. Choose \emph{terminal} as the type for the virtual
  machine, which generates the simplest start-up scripts for your
  virtual machine. The other types allow additional configuration
  scripts to be loaded by your virtual machine at start-up. Press the
  \emph{Add machine} button to add more virtual machines to the lab. 
\item[Network interfaces:] Every virtual machine you define should
  have at least one network interface. A default interface,
  \verb$eth0$, is available, but you must choose the so called
  \emph{collision domain} or LAN for that interface. These are
  typically upper case letters, A, B, C, etc, but can also be any
  string. Press the \emph{Add interface} button to add additional
  network interfaces which would be connected to different collision
  domains. Routers, for example, are expected to have more than one
  interface and be connected to more than one LAN.  Provide an IP
  address under field \emph{IP/Net} if you wish to have the interface
  ready to use once the virtual machine starts up.
\end{description}

\smallskip
To create a simple lab consisting of two virtual machines connected together
in a single Ethernet LAN, follow these steps:

\begin{enumerate}[label=(\arabic*)]
\item Start the KatharaGUI application. Define a lab with the
  description ``Lab 0''. Add two machines with names ``pc1'' and
  ``pc2'' with IP addresses ``10.0.0.1/24'' and ``10.0.0.2/24''
  respectively. This type of IP address is called \emph{CIDR} address
  (\url{https://searchnetworking.techtarget.com/definition/CIDR}). 
  Define the same collision domain, ex ``A'' for the interfaces of
  both machines. This makes sure both pc1 and pc2 are connected to the
  same LAN. 

\item Once the configuration is defined, click on the ``Graph'' button
  at the top of the KatharaGUI window to visualize your network (see
  Fig.~\ref*{2nodes.fig}). Press
  ``Download Lab'' to download a zip file with the necessary lab
  configuration files. Unpack the zip file to a sub-folder somewhere
  in your working directory, for example in a folder named
  ``Lab0''. You can inspect the generated files. They are text files.

\item Press ``Save/Load'' to export your lab configuration data to a
  file for later use.
\end{enumerate}

\insfig{2nodes}{.5}{A simple network configuration consisting of two
  hosts connected to the same LAN}


\subsection{Testing the first network}

Once the lab configuration file is defined, we can start \kathara\ and
experiment with our first network.

\begin{enumerate}[label=(\arabic*)]
\item Change your current directory in the folder where you unpacked
  the lab files. Run the \kathara\ command
  \verb$lstart$ (see Section \ref{examples.lab0} for additional
    information).

  \item Depending on the options passed when you started the lab, you
    should have access to two shell windows with user root, one on pc1
    and the other on pc2. Run \verb$ifconfig$ to check the IP address
    information on both hosts. Run \verb$ping$ to check for
    connectivity between the two hosts. See Section
    \ref{examples.lab0}.

  \item Unfortunately, command \verb$netcat$ (also known as \verb$nc$)
    is not available on the virtual machines, but it can be accessed
    from your home directory on the host computer, the computer that
    is running the network simulation. See Section \ref{examples.lab0}
    about how to execute netcat.

    Run netcat in listening mode on one of the hosts. From the other
    host, connect netcat to the instance you left listening.

  \item Use netcat to transfer a text file from one host to the
    other. 
\end{enumerate}



\subsection{Examples and other resources}\label{examples.lab0}

\begin{description}
\item[Configure \kathara:]
Replace the path to the bin folder in this example with your actual
path. You can insert this line in your \verb$.bash_rc$ file so that it
is executed every time you open a new terminal.
\begin{lstlisting}[language=bash]
  export NETKIT_HOME=/opt/kathara/bin/
\end{lstlisting}

\item[Start a \kathara\ lab using gnome-terminal as terminal instead
  of xterm:] You should consider using this command only if you do not
  like the terminal opened by default for every virtual host by
  \kathara. 
  Change current directory to the directory with the lab files. Then,
\begin{lstlisting}[language=bash]
  /opt/kathara/bin/lstart --xterm=gnome-terminal
\end{lstlisting}

\item[Stop a running \kathara\ lab:] This command is useful also when
  you cannot start a lab because \kathara\ reports that existing
  virtual hosts were detected and \kathara\ attempts to link terminals
  to these hosts. Most likely \kathara\ fails reusing those virtual
  hosts. Cleaning any trace of previous \kathara\ sessions will solve
  the problem.

  Change current directory to the
  directory with the lab files. Then, 
\begin{lstlisting}[language=bash]
  /opt/kathara/bin/lclean
\end{lstlisting}

\item[Start a \kathara\ lab without opening terminals for each virtual
  machine:] This operation requires root privileges on the host
  machine (the computer running the simulation), thus you can only run
  this on your personal computer. You might like this operation in
  case you only want to open terminals on specific virtual hosts, not
  on every host as is the default behaviour. 

  Start the lab using the \verb$--noterminals$ option. Then
  use Linux container management commands (the virtualisation
  technique that \kathara\ is built on) to find out the process ID of
  the simulated host you want to interact with. Start a shell on that
  process.

  %\begin{lstlisting}[language=bash]
  \begin{lstlisting}
$ /opt/kathara/bin/lstart --noterminals
========================= Starting Lab ==========================
lab_description:    two-hosts
lab_author:         RB
=================================================================
e995fdf1bd0480c407692465ddffdd3c1b06b0fe56a40c0c2424dbb1833536e7
Applying brctl patch to link e995fdf1bd04
52dca663fd726d8b43e0464b6414e976844814500e52bef2cc3127847b810e9a
c6d3d7b08aa2963040600128b99997ee890482c4e112df7715cdff1d5da7836a
net.ipv4.conf.all.rp_filter = 0
net.ipv4.conf.default.rp_filter = 0
net.ipv4.conf.eth0.rp_filter = 0
net.ipv4.conf.lo.rp_filter = 0
net.ipv4.conf.all.rp_filter = 0
net.ipv4.conf.default.rp_filter = 0
net.ipv4.conf.eth0.rp_filter = 0
net.ipv4.conf.lo.rp_filter = 0
$ rbenkocz@poppy: ~sudo docker ps
[sudo] password for rbenkocz: 
CONTAINER ID        IMAGE                 COMMAND             CREATED             STATUS              PORTS               NAMES
c6d3d7b08aa2        kathara/netkit_base   "/bin/bash"         21 seconds ago      Up 18 seconds                           netkit_1000_pc2
52dca663fd72        kathara/netkit_base   "/bin/bash"         23 seconds ago      Up 20 seconds                           netkit_1000_pc1

$  sudo docker exec -it 52dca663fd72 /bin/bash
\end{lstlisting}


  \item[Access your home directory from within host machines:] You can
    save files created from within the virtual hosts by writing them
    to \verb$/hosthome/$. Similarly, you can read any file from your
    home directory within each of the virtual hosts by reading the
    files from folder \verb$/hosthome/$.

  \item[Running netcat in a \kathara\ virtual machine:] Find out the
    location of netcat (nc) on the host machine. Then copy the
    executable file to your home directory. From within a virtual
    machine, execute netcat by invoking the executable copied to your
    host machine.
 
    \begin{lstlisting}[language=bash]
      $ type -a nc
nc is /home/rbenkocz/bin/nc
nc is /bin/nc
      cp /bin/nc ~
    \end{lstlisting}

    On an virtual machine, call nc via folder \verb$/hosthome$.
    \begin{lstlisting}[language=bash]
      /hosthome/nc -l 4444  # (on pc1)

      /hosthome/nc 10.0.0.1 4444 (on pc2)
    \end{lstlisting}


  \item[Configure interface \emph{eth0} with an IPv6 address:] ~

    \begin{lstlisting}
Script started on Tue Sep  3 06:13:56 2019
# ls
bin   dev  home      hostlab  lib64  mnt  proc  run   srv  tmp         usr
boot  etc  hosthome  lib      media  opt  root  sbin  sys  typescript  var
# sysctl -a | grep disable_ipv6
sysctl: reading key "net.ipv6.conf.all.stable_secret"
net.ipv6.conf.all.disable_ipv6 = 1
sysctl: reading key "net.ipv6.conf.default.stable_secret"
net.ipv6.conf.default.disable_ipv6 = 1
sysctl: reading key "net.ipv6.conf.eth0.stable_secret"net.ipv6.conf.eth0.disable_ipv6 = 1

sysctl: reading key "net.ipv6.conf.lo.stable_secret"net.ipv6.conf.lo.disable_ipv6 = 1

# sysctl -w net.ipv6.conf.eth0.disable_ipv6=0
net.ipv6.conf.eth0.disable_ipv6 = 0
# ifconfig eth0 inet6 add 2100::2/64
# ping6 2100::1
PING 2100::1(2100::1) 56 data bytes
64 bytes from 2100::1: icmp_seq=1 ttl=64 time=0.307 ms
64 bytes from 2100::1: icmp_seq=2 ttl=64 time=0.140 ms
64 bytes from 2100::1: icmp_seq=3 ttl=64 time=0.122 ms
64 bytes from 2100::1: icmp_seq=4 ttl=64 time=0.095 ms
^C
--- 2100::1 ping statistics ---
4 packets transmitted, 4 received, 0% packet loss, time 3059ms
rtt min/avg/max/mdev = 0.095/0.166/0.307/0.082 ms
    \end{lstlisting}

    
  \item[Shutting down a \kathara\ lab gracefully:] From within the
    directory containing the lab files, execute

    \begin{lstlisting}[language=bash]
      /opt/kathara/bin/lclean
    \end{lstlisting}

    
  \item [Shutting down a \kathara\ lab forced:] Execute

    \begin{lstlisting}[language=bash]
      /opt/kathara/bin/lwipe
    \end{lstlisting}

    
\end{description}


\section{Exercises}

\begin{enumerate}
\item\label{ipv6.ex} \kathara\ virtual machines can be configured with
  IPv4 addresses 
  when creating the lab files in KatharaGUI. In this exercise, you
  will configure IPv6 addresses to the two virtual hosts, manually,
  from within each host. Test the connection using \verb$ping6$
  command. Use netcat to transfer a text file from one host to the
  other using IPv6 protocols. Use \verb$script$ to capture your
  command session in a log file.

\item Create a lab with three virtual machines connected to a single
  LAN. Configure IPv6 addresses to all three machines. Use
  \verb$ping6$ to test the connectivity between the three hosts. Can
  you use netcat to relay the data sent from host 1 to host 3 via host
  2?
\end{enumerate}


%%%%%%%%%%%%%%%%$

\chapter{Lab 1: Capturing and examining packets }\label{tcpdump.se}

\section{Objectives}

\begin{itemize}[label=--]
  \item Students will generate lab files configured with IPv6
    addresses. 
\item Students will use the \emph{tcpdump} program to capture network
  packets. 
\item Students will examine captured packets with
  \emph{Wireshark} on their personal computers, or with an online
  network packet visualization tool such as
  \url{https://hpd.gasmi.net/}. 
\item Students will identify the protocols
  present in the captured packets. 
\item Students will examine the structure of packets. 
\item Students will explore the IPv4 Address Resolution Protocol and
  the corresponding protocol for IPv6 by examining the exchange of
  relevant packets.

\end{itemize}


\section{Lab activities}

\subsection{Generate lab files with IPv6 addresses for
  hosts}\label{2nodes-ipv6.se} 

The KatharaGUI application demonstrated in Chapter \ref{intro.se}
provides dedicated entries to assign IPV4 addresses to the virtual
hosts in the network. We will add appropriate commands to the lab
configuration page in KatharaGUI so that the virtual network is
automatically configured with IPv6 addresses as well.

\begin{itemize}[label=--]
\item Start KatharaGUI, and load the config file you created in
  Chapter \ref{intro.se}. You should see the configuration for a local
  network with two nodes, with IPv4 addresses allocated to the two
  nodes.
  \item Locate the ``Network interfaces'' tab for each of the two
    machines in the lab. This tab contains an edit box titled
    ``Directly in pc1(pc2).startup''. The text you enter in these edit
    boxes is inserted in the startup script being executed, on each
    of the virtual machines, when they are started. Add the two
    commands you used to configure IPv6 addresses manually for
    Exercise \ref{ipv6.ex} on page \pageref*{ipv6.ex}.
  \item Save the altered configuration file under a different
    \emph{lab~.config} file. Download the lab files and unpack them in
    your working folder in your home directory.
  \item Start the lab using command \verb$lstart$.
  \item Test your configuration by checking that the IPv6 addresses
    have been assigned to the network interfaces for each of the
    virtual machines (command \verb$ifconfig$) and test the connection
    using IPv6 protocols (command \verb$ping6$).
\end{itemize}

\subsection{Tcpdump on a simple local network}

\begin{itemize}[label=--]
\item Start the lab corresponding to the simple local network with
  IPv6 addresses you
  created in Section \ref{2nodes-ipv6.se}, if the lab is not already running. 
\item Execute command \verb$tcpdump$ to capture all packets arriving
  on the network interface \verb$eth0$ of host pc1. See Section
  \ref{examples.lab1} for an example that captures all packets and
  writes them to a file to be analysed later. You can also consult
  several of the many online resources on \verb$tcpdump$
  \cite{tcpdump,tcpdump2}. Make sure your packet capture file (pcap
  file) resides somewhere in your home directory and not on the
  virtual machine.
\item On the other host, run command \verb$ping$ to test the
  connectivity to the machine from where you capture
  packets. Interrupt \verb$ping$ and \verb$tcpdump$ with CTRL-C.
\item If you are running the simulation \textbf{on your own computer}, install
  \emph{Wireshark} (\url{https://www.wireshark.org/}). 
\item \textbf{If you are working on your own computer}, open the saved
  packet capture file in \emph{Wireshark}. On the \textbf{school
    computers}, upload your pcap file to
  \url{https://hpd.gasmi.net}. Wireshark and gasmi.net are utilities
  used in the examination of the structure of packets.
\item Wireshark and gasmi.net decompose each captured packet into
  several structures (or headers). Each header is created by a
  different network component from the \emph{protocol
    stack}\footnote{A protocol stack is a close approximation of the
    layered organization of the network software components. At the
    bottom of the protocol stack sit components providing link-layer
    type services, such as Ethernet. At the top of the stack are
    components providing application layer type services.} of the
  machine. 

  Which headers are common to all of the captured packets?

  Find three different types of network packets from your captured
  file containing a different combination of network headers. For each
  of the types, enumerate the headers present in the packets. Which
  headers originate from protocols at the \emph{top} of the protocol
  stack?  
\end{itemize}


%%%%%
\subsection{Examples and other resources}\label{examples.lab1}

\begin{description}
\item[Capture all packets and write them to a packet capture file:] ~

  Useful options:
  \begin{description}
    \item [-t] Suppress printing time stamps (not relevant for our
      simulated environment).
    \item [-s] Size of captured packets. 0=full packet captured.
    \item [-i interface] Which interface to capture from.
    \item [-w file] Write packets to file, to be examined by 3rd party
      tools like wireshark.
  \end{description}
  
\begin{lstlisting}
  root@pc1:/# tcpdump -i eth0 -s0 -t -w /hosthome/somelocation/somefile.pcap
tcpdump: listening on eth0, link-type EN10MB (Ethernet), capture size 262144 bytes
^C11 packets captured
11 packets received by filter
0 packets dropped by kernel
\end{lstlisting}
  
%%%
\item[Delete an entry corresponding to an address from the ARP
  table:]~

\begin{lstlisting}[language=bash]
      arp -d address
\end{lstlisting}

%%
\item[Manipulating the neighbours table in IPv6:]~

See \url{https://www.linuxtopia.org/online_books/network_administration_guides/Linux+IPv6-HOWTO/Linux_IPv6_HowTo_x1187.html}

\end{description}

\section{Exercises}

\begin{enumerate}
\item\label{arp.ex} Start the lab you configured in Section
  \ref*{2nodes-ipv6.se}. Start packet capture on one of the hosts. On
  the other host, run command \verb$arp$. If it provides any output,
  delete the IPv4 association from its table (see Section
  \ref*{examples.lab1}). Run command \verb$ping$ to reach the other
  host. Execute \verb$arp$ again. On the other host, stop the packet
  capture process. 

  Answer the following questions. 
  \begin{enumerate}[label=(\alph*)]
    \item What is the purpose of the ARP protocol?

    \item Examine the pcap file. Which hosts have placed an ARP
      request? List the packets containing the ARP request as decoded
      by your network packet examination utility.

    \item Examine the pcap file. Which hosts answer each ARP request?
      How is the ARP request delivered and what
      address is used?

    \item List the packets containing the ARP response to the
      requests. Identify the source Ethernet address and the source IP
      address from one of the ARP reply packets. At what offset from
      the start of the packet are these addresses stored? What is the
      \emph{byte order} (\cite{dordal2019introduction} Section 11.1.5)
      by which Ethernet and IP addresses are stored in
      these packets?
  \end{enumerate}

%%%
\item This exercise is equivalent to Exercise \ref*{arp.ex} but for
  IPv6. Start the lab you configured in Section
  \ref*{2nodes-ipv6.se}. 
\begin{itemize}[label=-]
  \item Run \verb$ifconfig$ and note the output on
  one of the hosts, then start packet capture on the same host. 
\item On the other host, run command \verb$ip -6 neigh show$
  \cite{arpipv6} and note
  the output. 
\item \verb$ping6$ the other host using the global address
  \verb$2100::x$.
\item Run the \verb$ip$ command once more and note the output.
\item Now \verb$ping6$ the other host using the \textbf{link local}
  IPv6 address you discovered from the output of \verb$ifconfig$.
\item Run the \verb$ip$ command and note the output.
\end{itemize}

Answer the following questions.
  \begin{enumerate}[label=(\alph*)]
    \item What is the protocol $X$ equivalent to ARP in the IPv6 protocol
      suite? 

    \item What is happening when you execute the sequence of
      commands from this exercise?

    \item Examine the pcap file. Which hosts have placed a protocol
      $X$ request? List the packets containing the $X$ request as
      decoded by your network packet examination utility. Is a request
      present for the link local IPV6 address too?

    \item Examine the pcap file. Which hosts answer each $X$ request?
      How is the request delivered and what
      address is used?

    \item List the packets containing the response to the protocol $X$
      requests. Identify the source Ethernet address and the source IP
      address from one of the protocol $X$ reply packets. At what offset from
      the start of the packet are these addresses stored? 
  \end{enumerate}

\end{enumerate}

%%%%%%%%%%%%%%%%%%%%%%%%%%%%%%5
\bibliographystyle{plain} \bibliography{labman}

\end{document}
