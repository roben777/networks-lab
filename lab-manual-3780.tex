\documentclass[12pt]{book}              
%\parindent0pt  \parskip10pt             % make block paragraphs
%\raggedright                            % do not right justify

\usepackage[margin=2cm]{geometry}
\usepackage{paralist}
%\usepackage{url}
\usepackage{siunitx}
\usepackage{listings}
\usepackage{graphicx} 
\usepackage{booktabs} % for much better looking tables
\usepackage{array} % for better arrays (eg matrices) in maths
\usepackage{verbatim} % adds environment for commenting out blocks of text & for
\usepackage{subcaption} % make it possible to include more than one captioned fi
\usepackage{float}
\usepackage[hyperfootnotes=false]{hyperref}
%\floatstyle{boxed} 
\restylefloat{figure}

\lstset{
  framexleftmargin=10mm, 
  tabsize=2,
  language=c, 
  frame=single,
  numbers=none,
  keepspaces,
  basicstyle=\footnotesize,
%  basicstyle=\small\ttfamily,
}

\usepackage{tcolorbox}
\usepackage{xcolor}

\usepackage[
    type={CC},
    modifier={by-sa},
    version={4.0},
]{doclicense}


\title{Computer Networks - Laboratory manual}    % Supply information
\author{Robert Benkoczi, University of Lethbridge}    %   for the title page.
\date{\today\\ version 1.01}  %   Use current date. 

%%%%%% macros %%%%%%5
\newcommand{\kathara}{Kathar\'a}

%%% inserting figures 
%%% arg 1: figure file (no directory)
%%% arg 2: scale factor
%%% arg 3: caption
%%% labels are #1.fig
\newcommand{\insfig}[3]{
\begin{figure}[tb]
\begin{center}
\includegraphics[scale=#2]{figs-manual/#1}
\caption{#3\label{#1.fig}}
\end{center}
\end{figure}
}

%%%%%%%%%%%%%%%%%%%%%%%%

%\renewcommand{\chaptername}{Activity}

% Note that book class by default is formatted to be printed back-to-back.
\begin{document}                        % End of preamble, start of text.
\frontmatter                            % only in book class (roman page #s)
\maketitle    
\doclicenseThis                      
\tableofcontents                        % Print table of contents

\mainmatter % arabic page nrs

\chapter{Lab 0: An Introduction to the Tools }\label{intro.se}

\section{Objectives}

\begin{itemize}[--]
\item Using the network emulator available freely at
  \url{kathara.org}, students will define a simple local network with
  two hosts and
  will use the netcat utility to transfer data between the hosts.

\item If the network is properly defined, the emulator will correctly
  configure IPv4 addresses for the two hosts. Using standard commands
  available for hosts running Unix based operating systems, the
  students will configure IPv6 addresses and will test the
  connectivity using the IPv6 network protocols.

%% \item Students will use the \emph{tcpdump} program to capture network
%%   packets. Students will examine captured packets with
%%   \emph{Wireshark} on their personal computers, or with an online
%%   network packet visualization tool such as
%%   \url{https://hpd.gasmi.net/}. Students will identify the protocols
%%   that participated in the creation of the captured packets. 
\end{itemize}


\section{Lab activities}
\subsection{Installing and configuring \emph{\kathara}}

This lab manual uses a network emulator called \emph{\kathara},
available at \url{kathara.org}. The lab activities described here make
specific use of \kathara\ commands, however, many of these
activities can be adjusted to work with any other network emulator or
even with real networks. 

\begin{tcolorbox}[title=Lab safety note]
  Students are cautioned to exercise common sense when attempting any
  of the commands described in this manual on real computer
  networks. Please consult the UofL policies concerning the use of the
  data network \cite{its-policies} and also be aware of the federal
  laws that govern the use of computers and data networks
  \cite{law-canada,law-us}. 
\end{tcolorbox}

\paragraph{Installing \kathara}

Perform this activity ONLY on your home computer. On the school
computers, \kathara\ is already installed.

The first step needed to carry out most of the activities described in
this lab manual is to install \kathara. The emulator is
software that simulates, using virtualisation technology, several
computers on a single \emph{host machine}. The software also simulates
Ethernet local area networks to which several simulated hosts can be
connected to. Students can open terminal windows with root access on
the simulated hosts and can use the standard utilities \emph{almost}
like in a real environment.

In this section of the lab, students will install \kathara\ on their
home computers, an optional step. More information
including a few examples of commands are provided in Section
\ref{examples.lab0}. The following sequence of operations is
suggested: 

\begin{enumerate}[(1)]
\item If only configuring an already installed system, go to paragraph
  \ref*{config.se}. You will need root access to the computer you are
  installing \kathara\ on. Please do not execute this step on the
  school network.

\item Download the installation files for \kathara\ from
  \url{https://github.com/KatharaFramework/Kathara/releases}. 
Download both the \verb$KatharaGUI$ package for your platform and the
source code package. The source code package contains additional
folders, for example the \verb$bin$ folder, needed to run the
emulator. 

\item Unpack both package files into a system folder, such as
\verb$/opt/kathara$. 

\item Follow the installation instructions from
\url{https://github.com/KatharaFramework/Kathara/wiki/Linux} for a
Linux installation or the appropriate instructions for your platform
available from the wiki page \cite{kathara-wiki}. 

\end{enumerate}

\paragraph{Configuring \kathara:} \label{config.se}

Define the environment variable \verb$NETKIT_HOME$ to point to the
\verb$bin/$ folder under \kathara's installation directory (step 6 in
the Linux installation instructions).


\subsection{Creating a first network}\label{2nodes.se}

To test \kathara\ and to begin the second part of this lab, we must
define the network to be simulated. This definition resides in a
collection of configuration files that contains the topology of the
network as well as configuration files loaded by the virtual nodes in
the network. We call these configuration files \emph{lab files}.

Lab files are most easily created using the KatharaGUI application,
strangely available under the systems folder you installed \kathara\ 
(for example \verb$/opt/kathara/$) 
or under \verb$/opt/kathara/GUI/$ in your CS network at school. 

The following fields should be filled:
\begin{description}
\item[Lab information:] it is a good idea to add a description, so you
  can distinguish the generated files later on. When submitting
  exercises for an assignment, be sure to enter your name in the
  author field.
\item[Machine informations:] Fill a name for the first machine
  (host). It is a good idea to provide unique names to the hosts you
  create. Choose \emph{terminal} as the type for the virtual
  machine, which generates the simplest startup scripts for your
  virtual machine. The other types allow additional configuration
  scripts to be loaded by your virtual machine at startup. Press the
  \emph{Add machine} button to add more virtual machines to the lab. 
\item[Network interfaces:] Every virtual machine you define should
  have at least one network interface. A default interface,
  \verb$eth0$, is available, but you must choose the so called
  \emph{collision domain} or LAN for that interface. These are
  typically upper case letters, A, B, C, etc, but can also be any
  string. Press the \emph{Add interface} button to add additional
  network interfaces which would be connected to different collision
  domains. Routers, for example, are expected to have more than one
  interface and be connected to more than one LAN.  Provide an IP
  address under field \emph{IP/Net} if you wish to have the interface
  ready to use once the virtual machine starts up.
\end{description}

\smallskip
To create a simple lab consisting of two virtual machines connected together
in a single Ethernet LAN, follow these steps:

\begin{enumerate}[(1)]
\item Start the KatharaGUI application. Define a lab with the
  description ``Lab 0''. Add two machines with names ``pc1'' and
  ``pc2'' with IP addresses ``10.0.0.1/24'' and ``10.0.0.2/24''
  respectively. This type of IP address is called \emph{CIDR} address
  (\url{https://searchnetworking.techtarget.com/definition/CIDR}). 
  Define the same collision domain, ex ``A'' for the interfaces of
  both machines. This makes sure both pc1 and pc2 are connected to the
  same LAN. 

\item Once the configuration is defined, click on the ``Graph'' button
  at the top of the KatharaGUI window to visualize your network (see
  Fig.~\ref*{2nodes.fig}). Press
  ``Download Lab'' to download a zip file with the necessary lab
  configuration files. Unpack the zip file to a sub-folder somewhere
  in your working directory, for example in a folder named
  ``Lab0''. You can inspect the generated files. They are text files.

\item Press ``Save/Load'' to export your lab configuration data to a
  file for later use.
\end{enumerate}

\insfig{2nodes}{.5}{A simple network configuration consisting of two
  hosts connected to the same LAN}


\subsection{Testing the first network}

Once the lab configuration file is defined, we can start \kathara\ and
experiment with our first network.

\begin{enumerate}[(1)]
\item Change your current directory in the folder where you unpacked
  the lab files. Run the \kathara\ command
  \verb$lstart$ (see Section \ref{examples.lab0} for additional
    information).

  \item Depending on the options passed when you started the lab, you
    should have access to two shell windows with user root, one on pc1
    and the other on pc2. Run \verb$ifconfig$ to check the IP address
    information on both hosts. Run \verb$ping$ to check for
    connectivity between the two hosts. See Section
    \ref{examples.lab0}.

  \item Unfortunately, command \verb$netcat$ (also known as \verb$nc$)
    is not available on the virtual machines, but it can be accessed
    from your home directory on the host computer, the computer that
    is running the network simulation. See Section \ref{examples.lab0}
    about how to execute netcat.

    Run netcat in listening mode on one of the hosts. From the other
    host, connect netcat to the instance you left listening.

  \item Use netcat to transfer a text file from one host to the
    other. 
\end{enumerate}



\subsection{Examples and other resources}\label{examples.lab0}

\begin{description}
\item[Configure \kathara:]
Replace the path to the bin folder in this example with your actual
path. You can insert this line in your \verb$.bash_rc$ file so that it
is executed every time you open a new terminal.
\begin{lstlisting}[language=bash]
  export NETKIT_HOME=/opt/kathara/bin/
\end{lstlisting}

\item[Start a \kathara\ lab using gnome-terminal as terminal instead
  of xterm:] You should consider using this command only if you do not
  like the terminal opened by default for every virtual host by
  \kathara. 
  Change current directory to the directory with the lab files. Then,
\begin{lstlisting}[language=bash]
  /opt/kathara/bin/lstart --xterm=gnome-terminal
\end{lstlisting}

\item[Stop a running \kathara\ lab:] This command is useful also when
  you cannot start a lab because \kathara\ reports that existing
  virtual hosts were detected and \kathara\ attempts to link terminals
  to these hosts. Most likely \kathara\ fails reusing those virtual
  hosts. Cleaning any trace of previous \kathara\ sessions will solve
  the problem.

  Change current directory to the
  directory with the lab files. Then, 
\begin{lstlisting}[language=bash]
  /opt/kathara/bin/lclean
\end{lstlisting}

\item[Start a \kathara\ lab without opening terminals for each virtual
  machine:] This operation requires root privileges on the host
  machine (the computer running the simulation), thus you can only run
  this on your personal computer. You might like this operation in
  case you only want to open terminals on specific virtual hosts, not
  on every host as is the default behaviour. 

  Start the lab using the \verb$--noterminals$ option. Then
  use Linux container management commands (the virtualisation
  technique that \kathara\ is built on) to find out the process ID of
  the simulated host you want to interact with. Start a shell on that
  process.

  %\begin{lstlisting}[language=bash]
  {\small
  \begin{verbatim}
$ /opt/kathara/bin/lstart --noterminals
========================= Starting Lab ==========================
lab_description:    two-hosts
lab_author:         RB
=================================================================
e995fdf1bd0480c407692465ddffdd3c1b06b0fe56a40c0c2424dbb1833536e7
Applying brctl patch to link e995fdf1bd04
52dca663fd726d8b43e0464b6414e976844814500e52bef2cc3127847b810e9a
c6d3d7b08aa2963040600128b99997ee890482c4e112df7715cdff1d5da7836a
net.ipv4.conf.all.rp_filter = 0
net.ipv4.conf.default.rp_filter = 0
net.ipv4.conf.eth0.rp_filter = 0
net.ipv4.conf.lo.rp_filter = 0
net.ipv4.conf.all.rp_filter = 0
net.ipv4.conf.default.rp_filter = 0
net.ipv4.conf.eth0.rp_filter = 0
net.ipv4.conf.lo.rp_filter = 0
$ rbenkocz@poppy: ~sudo docker ps
[sudo] password for rbenkocz: 
CONTAINER ID        IMAGE                 COMMAND             CREATED             STATUS              PORTS               NAMES
c6d3d7b08aa2        kathara/netkit_base   "/bin/bash"         21 seconds ago      Up 18 seconds                           netkit_1000_pc2
52dca663fd72        kathara/netkit_base   "/bin/bash"         23 seconds ago      Up 20 seconds                           netkit_1000_pc1

$  sudo docker exec -it 52dca663fd72 /bin/bash
\end{verbatim}
  }


  \item[Access your home directory from within host machines:] You can
    save files created from within the virtual hosts by writing them
    to \verb$/hosthome/$. Similarly, you can read any file from your
    home directory within each of the virtual hosts by reading the
    files from folder \verb$/hosthome/$.

  \item[Running netcat in a \kathara\ virtual machine:] Find out the
    location of netcat (nc) on the host machine. Then copy the
    executable file to your home directory. From within a virtual
    machine, execute netcat by invoking the executable copied to your
    host machine.
 
    \begin{lstlisting}[language=bash]
      $ type -a nc
nc is /home/rbenkocz/bin/nc
nc is /bin/nc
      cp /bin/nc ~
    \end{lstlisting}

    On an virtual machine, call nc via folder \verb$/hosthome$.
    \begin{lstlisting}[language=bash]
      /hosthome/nc -l 4444  # (on pc1)

      /hosthome/nc 10.0.0.1 4444 (on pc2)
    \end{lstlisting}


  \item[Configure interface \emph{eth0} with an IPv6 address:] ~
    %\begin{lstlisting}[language=txt]
    {\small
    \begin{verbatim}
Script started on Tue Sep  3 06:13:56 2019
# ls
bin   dev  home      hostlab  lib64  mnt  proc  run   srv  tmp         usr
boot  etc  hosthome  lib      media  opt  root  sbin  sys  typescript  var
# sysctl -a | grep disable_ipv6
sysctl: reading key "net.ipv6.conf.all.stable_secret"
net.ipv6.conf.all.disable_ipv6 = 1
sysctl: reading key "net.ipv6.conf.default.stable_secret"
net.ipv6.conf.default.disable_ipv6 = 1
sysctl: reading key "net.ipv6.conf.eth0.stable_secret"net.ipv6.conf.eth0.disable_ipv6 = 1

sysctl: reading key "net.ipv6.conf.lo.stable_secret"net.ipv6.conf.lo.disable_ipv6 = 1

# sysctl -w net.ipv6.conf.eth0.disable_ipv6=0
net.ipv6.conf.eth0.disable_ipv6 = 0
# ifconfig eth0 inet6 add 2100::2/64
# ping6 2100::1
PING 2100::1(2100::1) 56 data bytes
64 bytes from 2100::1: icmp_seq=1 ttl=64 time=0.307 ms
64 bytes from 2100::1: icmp_seq=2 ttl=64 time=0.140 ms
64 bytes from 2100::1: icmp_seq=3 ttl=64 time=0.122 ms
64 bytes from 2100::1: icmp_seq=4 ttl=64 time=0.095 ms
^C
--- 2100::1 ping statistics ---
4 packets transmitted, 4 received, 0% packet loss, time 3059ms
rtt min/avg/max/mdev = 0.095/0.166/0.307/0.082 ms
    \end{verbatim}
    }

    
  \item[Shutting down a \kathara\ lab gracefully:] From within the
    directory containing the lab files, execute

    \begin{lstlisting}[language=bash]
      /opt/kathara/bin/lclean
    \end{lstlisting}

    
  \item [Shutting down a \kathara\ lab forced:] Execute

    \begin{lstlisting}[language=bash]
      /opt/kathara/bin/lwipe
    \end{lstlisting}

    
\end{description}


\section{Exercises}

\begin{enumerate}[1.]
\item\label{ipv6.ex} \kathara\ virtual machines can be configured with
  IPv4 addresses 
  when creating the lab files in KatharaGUI. In this exercise, you
  will configure IPv6 addresses to the two virtual hosts, manually,
  from within each host. Test the connection using \verb$ping6$
  command. Use netcat to transfer a text file from one host to the
  other using IPv6 protocols. Use \verb$script$ to capture your
  command session in a log file.

\item Create a lab with three virtual machines connected to a single
  LAN. Configure IPv6 addresses to all three machines. Use
  \verb$ping6$ to test the connectivity between the three hosts. Can
  you use netcat to relay the data sent from host 1 to host 3 via host
  2?
\end{enumerate}


%%%%%%%%%%%%%%%%$

\chapter{Lab 1: Capturing and examining packets }\label{tcpdump.se}

\section{Objectives}

\begin{itemize}[--]
  \item Students will generate lab files configured with IPv6
    addresses. 
\item Students will use the \emph{tcpdump} program to capture network
  packets. 
\item Students will examine captured packets with
  \emph{Wireshark} on their personal computers, or with an online
  network packet visualization tool such as
  \url{https://hpd.gasmi.net/}. 
\item Students will identify the protocols
  present in the captured packets. 
\item Students will examine the structure of packets resulting from a
  simple file transfer using netcat. 
\item Students will explore the 
  IPv4 Address Resolution Protocol by examining the exchange of relevant
  packets. 

\end{itemize}


\section{Lab activities}

\subsection{Generate lab files with IPv6 addresses for
  hosts}\label{2nodes-ipv6.se} 

The KatharaGUI application demonstrated in Chapter \ref{intro.se}
provides dedicated entries to assign IPV4 addresses to the virtual
hosts in the network. We will add appropriate commands to the lab
configuration page in KatharaGUI so that the virtual network is
automatically configured with IPv6 addresses as well.

\begin{itemize}[--]
\item Start KatharaGUI, and load the config file you created in
  Chapter \ref{intro.se}. You should see the configuration for a local
  network with two nodes, with IPv4 addresses allocated to the two
  nodes.
  \item Locate the ``Network interfaces'' tab for each of the two
    machines in the lab. This tab contains an edit box titled
    ``Directly in pc1(pc2).startup''. The text you enter in these edit
    boxes is inserted in the startup script being executed, on each
    of the virtual machines, when they are started. Add the two
    commands you used to configure IPv6 addresses manually for
    Exercise \ref{ipv6.ex} on page \pageref*{ipv6.ex}.
  \item Save the altered configuration file under a different
    \emph{lab~.config} file. Download the lab files and unpack them in
    your working folder in your home directory.
  \item Start the lab using command \verb$lstart$.
  \item Test your configuration by checking that the IPv6 addresses
    have been assigned to the network interfaces for each of the
    virtual machines (command \verb$ifconfig$) and test the connection
    using IPv6 protocols (command \verb$ping6$).
\end{itemize}

\subsection{Tcpdump on a simple local network}

\begin{itemize}[--]
\item Start the lab corresponding to the simple local network with
  IPv6 addresses you
  created in Section \ref{2nodes-ipv6.se}, if the lab is not already running. 
\item Execute command \verb$tcpdump$ to capture any packets arriving on the
  network interface of host pc1. 
\end{itemize}


%%%%%
\subsection{Examples and other resources}\label{examples.lab1}



\section{Exercises}

\begin{enumerate}[1.]
\item
\end{enumerate}

%%%%%%%%%%%%%%%%%%%%%%%%%%%%%%5
\bibliographystyle{plain} \bibliography{labman}

\end{document}
